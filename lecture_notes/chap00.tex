\chapter{Introduction}



These lecture notes are based on Refs.~\cite{tanay2021action, tanay2021integrability, Cho:2019brd}
 which aim to give closed-form solutions to the spinning, eccentric
binary black hole dynamics at 1.5PN via two different equivalent ways: (i)
the standard way of integrating the Hamilton's equations and (ii) using action-angle variables.



The above papers assume certain level of familiarity
with the symplectic geometric approach to classical mechanics, the non-fulfillment
of which on the reader's part may make the papers appear esoteric.
The purpose of these lecture notes is to give the reader this prerequisite
 knowledge which the above papers assume on the reader's part. 

Although these notes are meant to be pedagogical 
in the exposition, but they
lack rigor. If the reader 
finds these notes to be incomplete or lacking rigor, they are welcome to
refer to the sources cited in these notes as well as the above papers.



We use two kinds of filled boxes in these lecture notes
\begin{itemize}
\item The boxes with an explicit label ``Box'': These are meant
to give the reader a reference to more advanced sources to supplement
the material discussed herein. 
\item The boxes without an explicit label ``Box'': These simply 
summarize the important points.
\end{itemize}


Accompanying the above three papers (Refs.~\cite{tanay2021action, tanay2021integrability, Cho:2019brd}), and these lecture notes
is a \textsc{Mathematica} package \cite{MMA1}. This package can 
do the following
\begin{itemize}
\item can perform numerical integration of the 1.5PN equations of motion (EOMs).
\item can implement the analytical solution presented in Ref.~\cite{Cho:2019brd} with
1PN effects included.
\item can implement the action-angle-based analytical solution presented in
Refs.~\cite{tanay2021action, tanay2021integrability}
\item can compute the frequencies (within the action-angles framework) 
of the 1.5PN spinning BBHs
\item can compute the Poisson bracket between any two quantities.
\end{itemize}














