\chapter{Integrable systems and action-angle variables}      \label{define_integrable_sys}

\section{Definitions}        \label{definition}

We will focus our attention on systems which possess a time-independent
Hamiltonian \\

\textbf{Hamiltonian systems:} A dynamical system possessing a 
Hamiltonian $H(\vv{p}, \vv{q})$ and whose EOMs are given 
via Hamilton's equations.
\begin{align}   
\dot{q_i} =  \frac{\partial H}{\partial p_i},~  ~~~~~~~~~~~~~
\dot{p_i} = - \frac{\partial H}{\partial q_i}, \\
\end{align}




\textbf{Canonical transformation:} For a Hamiltonian system with the
Hamiltonian $H(\vv{p}, \vv{q})$, a transformation $\vv{Q}(\vv{p}, \vv{q}),
\vv{P}(\vv{p}, \vv{q})$ is called canonical if Hamilton's equations in
the old coordinates imply Hamilton's equations in the new coordinates, i.e.
\begin{align}
& \dot{q_i} =  \frac{\partial H}{\partial p_i},~  ~~~~~~~~~~~~~
\dot{p_i} = - \frac{\partial H}{\partial q_i}    \\
\implies &  \dot{Q_i} =  \frac{\partial K}{\partial P_i},~  ~~~~~~~~~~~~~
\dot{P_i} = - \frac{\partial K}{\partial Q_i}, 
\end{align}
where $K (\vec{P}, \vec{Q}) =
   H(   \vv{p}(\vv{P}, \vv{Q}), \vv{q}(\vv{P}, \vv{Q}) )$. \\



\textbf{Integrable system and action-angle variables:} 
We will define integrable  systems and action-angle variables (AAVs)
in one shot.
For integrable systems, 
canonical transformation $( \vec{p}, \vec{q}) \leftrightarrow $ {$ (  \vec{\cal{J}} , \vec{\theta})$} exists such that {$H = H(\vec{\mathcal{J}})$}
(or rather $ \pd H/ \pd \theta_i = 0$)
and also
{$\{ \vec{p}, \vec{q} \}(\theta_i + 2 \pi)  = \{ \vec{p}, \vec{q} \}(\theta_i ) $}.
The first equation means that the Hamiltonian depends only on the actions 
and not the angles.
The last equation means that $\vv{p}$ and $\vv{q}$ are $2 \pi$-periodic
functions of $\theta_i$'s. $\mc{J}_i$'s and $\theta_i$'s are respectively called
the action and the angle variables.




\section{Elementary properties of action-angle variables}

The above definition of action-angle variables may seem ad-hoc 
but some very useful conclusions follow from this definition.
First note that due to 
 $( \vec{p}, \vec{q}) \leftrightarrow $ {$ (  \vec{\cal{J}} , \vec{\theta})$} 
 being a canonical transformation, the actions 
 $\vec{\cal{J}} $ act like the new momenta variable and the 
 angles $\vec{\theta}$ act like the new position variables.
 
 
 Writing Hamilton's equations in terms of these new momenta and
 positions (actions and angles), we get
\begin{align}              
\dot{\mathcal{J}}_{i}  &=   -  \frac{\partial H}{ \partial \theta_{i}}    =0    &&   \Longrightarrow \mathcal{J}_{i} \text { stay constant } ,  \label{AA_eqn_1}   \\ 
\dot{\theta}_{i}   &=   \frac{\partial H }{ \partial \mathcal{J}_{i}} \equiv \omega_{i}(\overrightarrow{\mathcal{J}})     && \Longrightarrow \theta_{i}=\omega_{i}(\overrightarrow{\mathcal{J}}) t. \label{AA_eqn_2} 
\end{align}
We have chosen to 
call $\partial H/ \partial \mathcal{J}_{i} $ the frequencies $\omega_i$'s
since they denote the linear rate of increase of the 
corresponding angles $\theta_i$'s. These frequencies $\omega_i$'s are 
constants too since they are functions of only the constant $\mc{J}_i$'s.



More interesting and useful conclusions follow. It's clear that we know what 
actions and angles are at any later time given their values at an initial time
(actions stay constants, angle increase linearly with time at a known rate).
Therefore, \textit{if we know how to switch back} from $(  \vec{\cal{J}} , \vec{\theta})$
to $(\vv{p}, \vv{q})$, then we can have $\vv{p}(t), \vv{q}(t)$ for any later time $t$,
that is to say, we can have the solution of the system.


We now summarize the above conclusions.\\
\begin{tcolorbox}
From the above definition of action-angle variables, it follows that
\begin{itemize}
\item Actions are constants.
\item Angles increase linearly with time at constant  rate $\omega_i  =  \partial H/ \partial \mathcal{J}_{i} $ .
\item Having action-angle variables can help us have the solution $\vec{p}(t), \vec{q}(t)$ of the system.
\end{itemize}
\end{tcolorbox}




\section{Liouville-Arnold theorem}


\subsection{Statement of the theorem}

Stated loosely, the Liouville-Arnold (LA) theorem  says that if a Hamiltonian system with
$2 n$ phase-space variables (positions and momenta), possesses $n$ constants of motion (including the
Hamiltonian)
which mutually commute among themselves, then the system is integrable and it possesses action-angle variables.





\hfill \break


\begin{definition}[label=def:A]
For  a more rigorous statement of the the theorem, along with its
proof, the reader is to referred to Chapter 11 of Ref.~\cite{fasano}.
\end{definition}

\hfill \break



\subsection{Application of the Liouville-Arnold theorem to the spinning BBH system}


To apply the LA theorem to the spinning BBH system, we need to determine
$2n$, the total number of positions and momenta. The total number of coordinate
appears to be 12; each vector $\vv{R}, \vv{P}, \vv{S}_1$ and $\vv{S}_2$
contributes 3 components. Despite that, $2n \neq 12$.
The reason we can't count all 12 components of vectors 
$\vv{R}, \vv{P}, \vv{S}_1$ and $\vv{S}_2$
is because when it comes to spins $\vv{S}_1$ and $\vv{S}_2$, 
it is not clear which components are positions and which are momenta. 
Remember, $2n$ is supposed to be the total number of positions and momenta.



To delineate the positions and momenta clearly in 
the vectors $\vv{R}, \vv{P}, \vv{S}_1$ and $\vv{S}_2$, 
we reproduce parts of
Eqs.~\eqref{C0-PBs_defined_1} and \eqref{C0-spin-PB} 
\begin{equation}
\left\{R_{i}, P_{j}\right\}=\delta_{ij} \quad \text { and }   \left\{\phi_{A}, S_{B}^{z}\right\}=\delta_{A B},    \label{canonical PB}
\end{equation}
which lets us see that $\phi_A$ (the azimuthal 
angle of $\vv{S}_A$) and $S^z_B$ (the $z$-component of 
$\vv{S}_B$) act like position and momentum, respectively.
 It is in this sense that we need to 
count positions and momenta to determine ``$2n$''
 for the application of the LA theorem.
Every spin vector thus contributes 2 positions-momenta.
Only 2 coordinates are needed to specify each spin
since spin magnitudes are constants: $\dot{S_A} = \pb{S_A, H} = 0$
(easily follows from Eqs.~\eqref{canonical PB} 
and \eqref{C0-PBs_defined_2}).
Therefore $2n$ for our spinning BBH
is $3 + 3+ 2+2 = 10$, which means that we require $10/2 = 5$ mutually commuting constants of motion
to establish integrability.


The PB between 
any two quantities
among $R^i, P_i, \phi^i_A, S^z_{iA}$ that falls outside the purview of Eqs.~\eqref{canonical PB} and \eqref{PBs_defined_2} is 0.


\newpage

\hfill \break


\begin{definition}[label=def:B]
Rigorously speaking,
integrability, action-angle variables and the LA theorem are built
on the foundations of symplectic geometry (a branch of differential geometry).
If the above process of counting the number of positions and momenta
for the  application of the LA theorem
seems shaky to the reader due to lack of rigor,
they are referred to 
\cite{jose, arnold, marsden_1, marsden_2} for an introduction to symplectic manifolds
and Darboux coordinates and the statement of the LA theorem against this mathematical backdrop.
\end{definition}

\hfill \break



For the spinning BBH, the five required 
commuting constants have been long-known \cite{Damour:2001tu}.
They are 
\begin{align}
H,~~~~~~~ J^2,~~~~~~~ L^2,~~~~~~~ J_z,~~~~~~~ \SeffL.    \label{CCs}
\end{align}
These
quantities have already been defined in Sec.~\ref{statement}.
Hence the 1.5PN spinning BBH is integrable and it possesses action-angle variables. 

















