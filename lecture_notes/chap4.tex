\chapter{Computation of the first four actions}    \label{chapter-5}





Recall from Sec.~\ref{C3-construct_AA} that action variables are given by
\begin{equation}
\mathcal{J}_{i}=\frac{1}{2 \pi} \oint_{\gamma_{i}} \vec{P} \cdot d \vec{Q} , 
\end{equation}
where the integral is performed over  a loop on the $n$-dimensional 
submanifold defined by constant values of the $n$ commuting constants.
One way to remain on this submanifold is to flow under any of the 
$n$ commuting constants, for if you flow under any of the commuting 
constants, the other commuting constants don't change. This is because
under the flow of $C_j$, $C_i$ changes as (using Eq.~\eqref{H-flow})
\begin{align}
\frac{d C_i}{d \lambda} =  \pb{C_i, C_j}   = 0 .
\end{align}



Owing to this observation, we will try to
form loops for action integration while flowing 
under the commuting constants in the following sections.



\section{Computation of $\mc{J}_1$}    \label{result_first_action}



Let's flow under one of the commuting constants $J^2$, square of the 
magnitude of the total angular momentum $\vv{J} = \vv{L} + \vv{S}_1 + \vv{S}_2$.
Its flow equation is (with $\vv{V}$ standing for 
the column vector containing all the variables in $\vv{R}, \vv{P}, \vv{S}_1$
and $\vv{S}_2$)
\begin{equation}
\frac{d \vec{V}}{d \lambda} = 2 \vec{J} \times \vec{V},      \label{flow_action_1}
\end{equation}
which implies that all the four 3D vectors rotate around 
the fixed $2 \vv{J}$ vector (note that $\pb{\vv{J}, J^2} = 0$).
Using $\vv{n}$ to denote the vector around which all the four
3D vectors rotate, we see that $\vv{n} = 2 \vv{J} $.
In fact we had already worked this out and the pictorial 
representation has already been presented in the form 
of Fig.~\ref{Jsq_flow}.


Now, Eq.~\eqref{flow_action_1} implies that we will
arrive at where we started from (thus closing  a loop) after we have
flowed by an amount $\Delta \lambda = 
2 \pi/|\text{angular~velocity}|  
= 2 \pi/(2 J)   = \pi/J$.


Now, we break down the action integral as
\begin{equation}
\begin{aligned}
\mathcal{J} &=\mathcal{J}^{\text {orb }}+\mathcal{J}^{\text {spin }} \\
\mathcal{J}^{\text {orb }} & \equiv \frac{1}{2 \pi} \oint_{\mathcal{C}} \sum_{i} P_{i} d R^{i}   \\
\mathcal{J}_{A}^{\mathrm{spin}}  & =\frac{1}{2 \pi} \oint S_{A}^{z} d \phi_{A} .
\end{aligned}
\end{equation}
Let's tackle the orbital sector first. The orbital contribution
to the action integral becomes
\begin{align}
\mathcal{J}^{\text {orb }} &=\frac{1}{2 \pi} \int_{0}^{\Delta \lambda} P_{i} \frac{d R^{i}}{d \lambda} d \lambda=\frac{1}{2 \pi} \int_{0}^{\Delta \lambda} \vec{P} \cdot(\vec{n} \times \vec{R}) d \lambda \\
&=\frac{1}{2 \pi} \int_{0}^{\Delta \lambda} \vec{n} \cdot \vec{L} d \lambda=\hat{n} \cdot \vec{L}   .     \label{orbital_contribution}
\end{align}
The spin sector integral is 
\begin{align}
\mathcal{J}_{A}^{\mathrm{spin}}  & =\frac{1}{2 \pi} \oint S_{A}^{z} d \phi_{A},
\end{align}
which does not appear to be SO(3) covariant, but it actually is. 
This means we can that this integral is insensitive to
the rigid rotations of our coordinate axes.


\hfill \break


\begin{definition}[label=def:D]
Using the language of symplectic forms and differential geometric version 
of the generalized Stokes' theorem, we can see that 
$ \oint S_z d \phi =  \int d S_z \wedge d\phi$, the integral on the LHS is a
line integral, whereas that on the RHS is an area integral on the spin
sphere. Area integrals are indeed SO(3) covariant.
\end{definition}

\hfill \break


So, we rotate our axes so that the $z$-axis points along $\vv{n}$.
The spin sector integral is 
\begin{align}
\mathcal{J}_{A}^{\mathrm{spin}}   =\frac{1}{2 \pi} \oint S_{A}^{z} d \phi_{A} 
  & =    S_{A}^{z}   =  \hat{n} \cdot \vv{S}_A.   \label{spin_contribution}
\end{align}
We have used the fact that $S_{A}^{z}$ is constant on the loop of integration;
this is so because $\vv{S}_A$ makes a constant angle with the $z$-axis 
(or $\vv{n}$ vector) while we perform the line integral.



Finally, combining Eqs.~\eqref{orbital_contribution} and \eqref{spin_contribution},
our action integral becomes 
\begin{equation}
\mathcal{J}=\hat{n} \cdot\left(\vec{L}+\vec{S}_{1}+\vec{S}_{2}\right)=\hat{n} \cdot \vec{J} = J.
\end{equation}




\section{Computation of $\mc{J}_2$ and $\mc{J}_3$}

The procedure for computing the next two actions is very similar.
Instead of flowing under $J^2$, we flow under $L^2$ and $J_z$,
with the corresponding $\vec{n}$ being $2 \vv{L}$ and $\hat{z}$,
with the exception that under the $L^2$ flow, the spin vectors
don't move; only orbital ones do.
The amount of flow required to close the loop is still given by
$\Delta \lambda = 2 \pi/n$.


Doing similar calculations as above, we find that the 
corresponding actions turn out to be $\hat{n} \cdot \vec{J}$
which gives us 
\begin{equation}
\mathcal{J}_{2}= L , \quad \mathcal{J}_{3}=  J_z.
\end{equation}
All in all we finally have 
\begin{equation}
\mathcal{J}_{2}= J, \quad\mathcal{J}_{2}= L , \quad \mathcal{J}_{3}=  J_z.
\end{equation} \\



\begin{tcolorbox}
From the expressions of the above three actions, we note two features
that actions appear to possess
\begin{itemize}
\item Action variables are functions of the commuting constants: $\mc{J}(\vec{C})$.
In fact, all actions are constants and are also mutually commuting. 
\item An action is a function of only those $C_i$'s under which we need to flow
to close the loop, the integral over which furnishes the action. 
\end{itemize}
\end{tcolorbox}





\section{Computation of $\mc{J}_4$}


We won't derive $\mc{J}_4$ because, this action also
has a Newtonian limit which is derived in graduate level texts; see
Eq.~(10.139) of Ref.~\cite{goldstein2013classical}. 
It's 1PN extension was worked out in Ref.~\cite{Damour:1988mr}, 
see Eq.~(3.10) therein. The 1.5PN version of this action is given 
in Eq.~(38) of Ref.~\cite{tanay2021integrability}.
The methods to achieve these PN versions of the fourth action
is similar and chiefly involves complex contour integration technique
invented by Arnold Sommerfeld.



For the reference of the reader, the fourth action is
\begin{equation}
\mathcal{J}_{4}=-L+\frac{G M \mu^{3 / 2}}{\sqrt{-2 H}}+\frac{G M}{c^{2}}\left[\frac{3 G M \mu^{2}}{L}+\frac{\sqrt{-H} \mu^{1 / 2}(\nu-15)}{\sqrt{32}}-\frac{2 G \mu^{3}}{L^{3}} \vec{S}_{\mathrm{eff}} \cdot \vec{L}\right]+\mathcal{O}\left(c^{-4}\right)
\end{equation}


















